\begin{center}
\scalebox{0.7}
{
    \begin{circuitikz}[thick]
    % Ersatzschaltung für einen Reellen Kondensator
    % 
    % Author:   Alex Krieg
    % Date:     04.06.2024

    
    % Gitternetzlinien im Hintergrund
    %\draw[xstep=1cm, ystep=1cm, line width=0.1mm, color=lightgray] (0,0) grid (6,4);

    % Einstellungen für Symbole
    \ctikzset
    {  
        inductor            =   american, 
        resistor            =   european,
        inductors/scale     =   1.0, 
        capacitors/scale    =   0.8,
        diodes/scale        =   0.6,
        line width          =   0.5,
    }
    \def\labelOffset{0.4}

    % Knoten
    \coordinate (input) at (0,2);
    \coordinate (K1) at (2,2);
    \coordinate (K2) at (4,2);
    \coordinate (output) at (6,2);
    \coordinate (K3) at (2,1);
    \coordinate (K4) at (4,1);
    \coordinate (K5) at (6,2);

    % Kreuzungspunkte
    \draw (input)   node[circ]{};
    \draw (K1)      node[circ]{};
    \draw (K2)      node[circ]{};
    \draw (output)  node[circ]{};

    % Schaltung
    %      Start Pos    Symbol type             Name    End Pos
    \draw (input)   to [L,                      l=ESL]  (K1);             % Suple
    \draw (K3)      to [R,                      l=$R_{\rm leak}$]  (K4);        % Widerstand (leakage)
    \draw (K1)      to [capacitor,              l=$C$]  (K2);               % Kondensator
    \draw (K2)      to [R,                      l=ESR]  (K5);           % Widerstand (serie)
    \draw (K1)      to [short](K3);
    \draw (K2)      to [short](K4);

  
    \draw (K5)      to [short](output);
    \end{circuitikz}
}
\end{center}

